\subsection{i-IDRM}
\label{sec::iidrm}

\begin{algorithm}[htb!]
\floatname{algorithm}{Pseudo code}
\caption{Periodic intra-domain rountines}
\label{alg:intratimer}
{\small
\begin{algorithmic}[1]
\IF {(timer $>$ i-IDRM Beacon interval)}
	\STATE send i-IDRM beacons to $G(D(i))$
\ENDIF
\end{algorithmic} 
}
\end{algorithm} 

Pseudo code \ref{alg:intratimer} is a simple routine to send i-IDRM beacons
periodically to gateways within a MANET.
The main propose of i-IDRM beacon is to maintain $G^{\sf intra}(i)$ at different
gateways, which will be used in pseudo code \ref{alg:intraMain} to generate new
MANET ID when partitioned or merged happen.

\begin{algorithm}[htb!]
\floatname{algorithm}{Pseudo code}
\caption{Main intra-domain rountine}
\label{alg:intraMain}
{\small
\begin{algorithmic}[1]
\IF {(recv i-IDRM Beacon)}
\STATE update $G^{\sf intra}(i)$
	\IF {($G^{\sf intra}(i)$ changed)}
	\STATE generated new MANET ID
	\STATE invoke e-IDRM to send route update (in pseudo code \ref{alg:interMain})
	\ENDIF
\ENDIF
\STATE
\IF {(recv i-IDRM route update packet)}
	\STATE update IDRM route table according to MANET policy
	\IF {(route table changed)}
		\STATE invoke e-IDRM to send route update
	\ENDIF 
\ENDIF
\STATE
\IF {(recv e-IDRM route update from e-IDRM)}
	\STATE send i-IDRM route update to gateways in $G^{\sf intra}(i)$
\ENDIF 
\STATE
\IF {(recv data packet)}
	\STATE handleDataPacket(pkt)  
\ENDIF
\end{algorithmic} 
}
\end{algorithm} 

Pseudo code \ref{alg:intraMain} is the main routine for i-IDRM. 
The main functions of this routine is to
1) maintain $G^{\sf intra}(i)$, and
2) exchange route update between  $G^{\sf intra}(i)$.
As we describe in section \ref{x}, $G^{\sf intra}(i)$ is used to generate new
MANET ID when a domain is partitioned or merged. 
Once a new MANET ID is generated, 
a route update will be sent to $G^{\sf inter}(i)$.
When i-IDRM receives an i-IDRM route update 
from other nodes in $G^{\sf intra}(i)$, 
it will update the IDRM route table and 
notify e-IDRM to propogate any updates to inter-domain gateways. 
In i-IDRM, any updates at IDRM route table will not send to $G^{\sf intra}(i)$
because other gateways in $G^{\sf intra}(i)$ will also received 
the route update message from the update originator. 
On the other hand, when i-IDRM receives an e-IDRM route update,
i-IDRM will help e-IDRM to distribute the route update to  $G^{\sf intra}(i)$.

 