\subsection{Protocol Specification}
\label{sec::protocol}

This section describes the inter-domain routing protocol of IDRM
in pseudo codes. We present three algorithms to be executed at each
gateway. Algorithm 1 is a subroutine to generate route updates
(including route announcement and withdrawal). Algorithm 2 is a continual process of
a gateway to handle the interaction between inter-domain
gateways. Algorithm 3 is a continual process to manage the
intra-domain membership.

For a gateway $i$ in a domain $A$, let $G^{\sf intra}(i)$ denote a set
consisting of the intra-domain gateways to that $i$ has connectivity,
and $G^{\sf inter}(i)$ denote a set consisting of the inter-domain
gateways $i$ is directly connected. Let $M(i)$ denote the set of
intra-domain members to that $i$ has connectivity.


\begin{algorithm}[htb!]
\caption{Route Announcement Update}
\label{alg:announce}
{\small
\begin{algorithmic}
\STATE need\_update $\leftarrow$ FALSE
\STATE // store old route annoucement for withdrawal
\IF{([MD, {\bf path}] $\ne$ NULL)}
\STATE withdraw[MD, {\bf path}] $\leftarrow$ [MD, {\bf path}]
\ENDIF
\IF{(any change in $G^{\sf intra}(i)$)}
\STATE // generate a new MANET\_ID
\STATE MANET\_ID $\leftarrow f(A, G^{\sf intra}(i))$
\STATE need\_update $\leftarrow$ TRUE 
\STATE // else MANET\_ID does not change
\ENDIF
\IF{(any change in $M(i)$)} 
\STATE // generate a new membership digest
\STATE MD $\leftarrow b(M(i))$
\STATE need\_update $\leftarrow$ TRUE 
\STATE // else the membership digest does not change
\ENDIF
\STATE {\bf path} $\leftarrow$ \{MANET\_ID\}
\STATE return a new route announcement [MD, {\bf path}]
\end{algorithmic} 
}
\end{algorithm} 


Algorithm 1 checks any change in the membership of $G^{\sf intra}(i)$
and $M(i)$, and generate a new route announcement if necessary, and a
route withdrawal that uses old membership information and MANET
ID. Here, the function $f$ denotes a one-way hash function (e.g., MD5)
to create a MANET\_ID based on the original domain ID, and the set of
gateways, and the membership digest based on $M(i)$.
%%(seeSection~\ref{sec::membership} for how to obtain $M(i)$)
%Function $b$ can be configured to either return a plain membership list or a membership summary (e.g., using a Bloom filter).
%% For secure communications of route
%% announcements, we can use keyed hash functions for $H$ and
%% synchronously update the keys periodically so that unauthorized node
%% cannot generate a fake membership digest. 
%[Starsky: removed as not related]
%One practical concern is securing the route updates to prevent attacks
% on routing table. Securing the route updates and other aspects of
% path vector protocol in MANETs is a future research topic.
% %which will be briefly discussed in Section~\ref{sec::discussion}.

\begin{algorithm}[htb!]
\caption{Main Routine of the Gateway}
\label{alg:newconn}
{\small
\begin{algorithmic}
\WHILE{(true)}
\IF {(timer $>$ announcement interval)}
\STATE // generate a new route announcement
\STATE call Algorithm 1
\IF  {need\_update}
\IF  {withdraw[MD, {\bf path}] $\ne$ NULL}
\STATE send withdrawal withdraw[MD, {\bf path}] to $G^{\sf inter}(i)$
\STATE withdraw[MD, {\bf path}]  $\leftarrow$ NULL
\ENDIF
\STATE send route announcement [MD, {\bf path}] to $G^{\sf inter}(i)$
\ENDIF
\ENDIF
\STATE // propagate route withdrawal 
\IF {(received a route withdrawal withdraw[MD, {\bf path}])}
\STATE // update the path vector
\STATE delete [MD, {\bf path}]
\STATE {\bf path} $\leftarrow$ append (MANET\_ID, {\bf path})
\STATE announce withdraw[MD, {\bf path}] to $G^{\sf inter}(i) \cup G^{\sf intra}(i)$
\ENDIF
\STATE // propagate route announcement 
\IF {(received a route announcement [MD, {\bf path}])}
\IF {(announcement from $g^{\sf new}$ not in $G^{\sf inter}(i)$)}
\STATE // new connected inter-domain gateway found
\STATE $G^{\sf inter}(i) \leftarrow G^{\sf inter}(i) \cup \{ g^{\sf new} \}$
\ENDIF
%\IF {(route in {\bf path} is allowed by local policy)}
\IF {((no route to MD) OR ({\bf path} $\prec$ route to MD))}
\STATE // update the path vector
\STATE insert [MD, {\bf path}] at the top
\STATE {\bf path} $\leftarrow$ append (MANET\_ID, {\bf path})
\STATE announce [MD, {\bf path}] to $G^{\sf inter}(i) \cup G^{\sf intra}(i)$
\ENDIF
\ENDIF
%\ENDIF
\STATE increment timer and sleep
\ENDWHILE
\end{algorithmic}  
}
\end{algorithm} 



Algorithm 2 presents the main function of a gateway participating in
IDRM. The main routine consists of two parts. First, it periodically
polls its domain status, generates a new route announcement or route
withdrawal, and broadcasts the route updates to its neighbouring
inter-domain gateways. Second, it wakes up when a new route withdrawal
or announcement is received from one of its neighbours and process
them. In the route announcement, {\bf path} is an ordered list of
MANET\_IDs, i.e., [MANET\_ID$_1$, $\ldots$, MANET\_ID$_n$], which
indicates the nodes in MD can be reached by traversing MANET\_ID$_1$,
then MANET\_ID$_2$, $\ldots$, and finally MANET\_ID$_n$.  When it
processes a route withdrawal, it first deletes the path vector as
indicated by route withdrawal, and then propagate the route withdrawal
by appending its MANET ID.  When it processes a route announcement, it
first examines if the origin of the announcement is already in its
list of neighbours. If it is a new neighbour it updates the list. Then
it compares the new path information using its inter-domain routing
policy. If the route specified in the {\bf path} is allowed and is
more preferable than the current route to MD based on inter-domain
routing policy (i.e., {\bf path} $\prec$ route to MD), the gateway
updates its routing table by inserting the new route to the top
(assuming that the preference of a route is determined by the order in
the routing table), and then appends its own MANET ID in front of {\bf
path} and rebroadcasts the information to its neighbours.



\begin{algorithm}[htb!] 
\caption{Beaconing among Intra-domain Gateways}
\label{alg2}
{\small
\begin{algorithmic}
%\STATE {\em Initialization}: Endowed with $G^{\sf intra}(i)$
\WHILE{(true)}
\IF{(timer $>$ beacon interval)}
\STATE send beacons to every gateway in $G^{\sf intra}(i)$
\ENDIF
%\STATE set {\sf Connected\_Gateways} = empty
\FORALL {(gateway $g$ in $G^{\sf intra}(i)$)}
\IF {(no beacons from $g$ within time limit)}
\STATE // network has partitioned
\STATE $G^{\sf intra}(i) \leftarrow G^{\sf intra}(i) \backslash \{ g \}$
\STATE raise change flag
%\STATE call {\bf Algorithm} 1, generate route announcement; 
\ENDIF
\ENDFOR
\IF{(received a beacon from $g$ not in $G^{\sf intra}(i)$)}
\STATE // network merge event OR new gateway
%\STATE forward the beacon from $g$ to $G^{\sf intra}(i)$
\STATE $G^{\sf intra}(i) \leftarrow G^{\sf intra}(i) \cup \{ g \}$
\STATE raise change flag
%\STATE call {\bf Algorithm} 1, generate route announcement; 
\ENDIF
\IF{(change flag is up)}
\STATE // generate a new route announcement 
\STATE call Algorithm 1
\STATE send the route announcement to $G^{\sf inter}(i)$
\STATE reset change flag 
\ENDIF
\STATE increment timer and sleep
\ENDWHILE
\end{algorithmic} 
} 
\end{algorithm} %




Algorithm 3 is a separate thread that takes care of the exchange of
beacons among the gateways in the same domain. Periodically, a gateway
sends out a beacon to all intra-domain gateways notifying its
presence. When it does not receive a beacon from one or more of the
gateways in its intra-domain, it updates $G^{\sf intra}(i)$. Similarly, when it
receives a beacon from a gateway $g$ that is not currently in the list
of intra-domain gateways, it updates its entry. When these changes are
detected, the gateway initiates a route announcement process to update
its neighbours.


\input{eIDRM}
\subsection{i-IDRM}
\label{sec::iidrm}

\begin{algorithm}[htb!]
\floatname{algorithm}{Pseudo code}
\caption{Periodic intra-domain rountines}
\label{alg:intratimer}
{\small
\begin{algorithmic}[1]
\IF {(timer $>$ i-IDRM Beacon interval)}
	\STATE send i-IDRM beacons to $G(D(i))$
\ENDIF
\end{algorithmic} 
}
\end{algorithm} 

Pseudo code \ref{alg:intratimer} is a simple routine to send i-IDRM beacons
periodically to gateways within a MANET.
The main propose of i-IDRM beacon is to maintain $G^{\sf intra}(i)$ at different
gateways, which will be used in pseudo code \ref{alg:intraMain} to generate new
MANET ID when partitioned or merged happen.

\begin{algorithm}[htb!]
\floatname{algorithm}{Pseudo code}
\caption{Main intra-domain rountine}
\label{alg:intraMain}
{\small
\begin{algorithmic}[1]
\IF {(recv i-IDRM Beacon)}
\STATE update $G^{\sf intra}(i)$
	\IF {($G^{\sf intra}(i)$ changed)}
	\STATE generated new MANET ID
	\STATE invoke e-IDRM to send route update (in pseudo code \ref{alg:interMain})
	\ENDIF
\ENDIF
\STATE
\IF {(recv i-IDRM route update packet)}
	\STATE update IDRM route table according to MANET policy
	\IF {(route table changed)}
		\STATE invoke e-IDRM to send route update
	\ENDIF 
\ENDIF
\STATE
\IF {(recv e-IDRM route update from e-IDRM)}
	\STATE send i-IDRM route update to gateways in $G^{\sf intra}(i)$
\ENDIF 
\STATE
\IF {(recv data packet)}
	\STATE handleDataPacket(pkt)  
\ENDIF
\end{algorithmic} 
}
\end{algorithm} 

Pseudo code \ref{alg:intraMain} is the main routine for i-IDRM. 
The main functions of this routine is to
1) maintain $G^{\sf intra}(i)$, and
2) exchange route update between  $G^{\sf intra}(i)$.
As we describe in section \ref{x}, $G^{\sf intra}(i)$ is used to generate new
MANET ID when a domain is partitioned or merged. 
Once a new MANET ID is generated, 
a route update will be sent to $G^{\sf inter}(i)$.
When i-IDRM receives an i-IDRM route update 
from other nodes in $G^{\sf intra}(i)$, 
it will update the IDRM route table and 
notify e-IDRM to propogate any updates to inter-domain gateways. 
In i-IDRM, any updates at IDRM route table will not send to $G^{\sf intra}(i)$
because other gateways in $G^{\sf intra}(i)$ will also received 
the route update message from the update originator. 
On the other hand, when i-IDRM receives an e-IDRM route update,
i-IDRM will help e-IDRM to distribute the route update to  $G^{\sf intra}(i)$.

 
