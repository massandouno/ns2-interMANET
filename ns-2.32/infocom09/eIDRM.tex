\section{Protocol Specification}
\label{sec::specification}

In this section we describe the routing protocol of IDRM in pseudo codes. 
We divide the functionalities of IDRM into two sub-protocols, namely external
IDRM (e-IDRM) and internal IDRM (i-IDRM). 
The main duty of e-IDRM is to coordinate communication between gateways in
different MANETs and perform route maintenance, 
while i-IDRM is used to handle communication between gateways within
a single MANET.

For a gateway $i$, let $D(i)$ denote the orginial Domain ID of $i$,
and let $G(D(i))$ denote the set of the intra-domain gateways in domain $D(i)$.
Also, $G^{\sf intra}(i)$ denote the set of the intra-domain gateways
which $i$ has connectivity,
and $G^{\sf inter}(i)$ denote the set of the inter-domain
gateways which are within $i$'s single hop range. 
Let $M(i)$ denote the set of intra-domain members to that $i$ has connectivity.

\subsection{e-IDRM}
\label{sec::eidrm}  


\begin{algorithm}[htb!]
\floatname{algorithm}{Pseudo Code}
\caption{Periodic inter-domain rountines}
\label{alg:intertimer}
{\small
\begin{algorithmic}[1]
\IF {(timer $>$ e-IDRM Beacon interval)}
	\STATE broadcast e-IDRM beacon
\ENDIF
\STATE
\IF {(timer $>$ route maintenance interval)}
	\FOR{all routes in route table}  
		\STATE remove expired route
	\ENDFOR
	\STATE read native ad-hoc routing table and update $M(i)$
	\STATE update IDRM route table according to $M(i)$
	\STATE
	\IF {(route table changed)}
		\STATE broadcast e-IDRM route update
		\STATE invoke i-IDRM to send route update (in pseudo code
		\ref{alg:intraMain})
	\ENDIF
\ENDIF

\end{algorithmic} 
}
\end{algorithm} 

Pseudo code \ref{alg:intertimer} is executed periodically to keep 
1) broadcasting e-IDRM beacons to neighbour inter-domain gateways,
2) removing expired route entries in IDRM route table, 
3) reading underlying ad-hoc routing table and update $M(i)$ and
IDRM route table accordingly, and 
4) sending route update information to other gateways if there are any route
changes. 
The e-IDRM beacons are sent by broadcast so that new joining gateways
can learn the neighbouring information immediately.
For route maintenance, when there are changes in the IDRM route table,
the gateway will send a route update message to $G^{\sf inter}(i)$ and
rely on i-IDRM to distribute the route update information to 
$G^{\sf intra}(i)$. 
A route update message can consist of the combination of
following information: 1) new route is added, 
2) existing route is updated, 
3) existing route is deleted, and 
4) new MANET ID is generated. 
Notice that IDRM depends on the underlying ad-hoc routing protocol to provide
$M(i)$, in other words, gateway does not probe the membership by itself. 

\begin{algorithm}[htb!]
\floatname{algorithm}{Pseudo Code}
\caption{Main inter-domain rountine}
\label{alg:interMain}
{\small
\begin{algorithmic}[1]
\IF {(recv e-IDRM Beacon)}
	\STATE update $G^{\sf inter}(i)$ according to MANET policy
	\STATE update IDRM route table according to MANET policy
	\STATE
	\IF {(new gateways are added to $G^{\sf inter}(i)$)}
		\STATE exchange IDRM route table with the new gateways
	\ENDIF
	\STATE
	\IF {(route table changed)}
		\STATE broadcast e-IDRM route update
		\STATE invoke i-IDRM to send route update
	\ENDIF 
\ENDIF

\STATE 
\IF {(recv e-IDRM route update packet)}
	\STATE update IDRM route table according to MANET policy
	\STATE
	\IF {(route table changed)}
		\STATE broadcast e-IDRM route update
		\STATE invoke i-IDRM to send route update
	\ENDIF 
\ENDIF
\STATE
\IF {(recv i-IDRM route update packet)}
	\STATE broadcast e-IDRM route update
\ENDIF 

\STATE
\IF {(recv data packet)}
	\STATE handleDataPacket(pkt)
\ENDIF
\end{algorithmic} 
}
\end{algorithm} 

Pseudo code \ref{alg:interMain} is the main routine which consists of the core
e-IDRM funtions. 
When a gateway recevies an e-IDRM beacon or a e-IDRM route update,
it will update its IDRM route table. 
In here, a gateway detects a new inter-domain gateway, 
they will first exchange their IDRM route tables.
Similar to pseudo code \ref{alg:intertimer}, 
route update messages will be sent to $G^{\sf inter}(i)$ 
when there are route changes, 
and propagation of the route update information for intra-domain gateway
will rely on i-IDRM protocol.
%The propagation of route changes to intra-domain
%gateways will be handled by i-IDRM in pseudo code \ref{alg:intraMain}.
When the gateway receives a data packets, 
it will pass it to Function HandleDataPacket() 
in pseudo code \ref{alg:handleDatapacket}.
 
\begin{algorithm}[htb!]
\floatname{algorithm}{Pseudo Code}
\caption{Function HandleDataPacket(pkt)}
\label{alg:handleDatapacket}
{\small
\begin{algorithmic}[1]
	\IF {(destination is listed in native ad-hoc route table)}
		\STATE forward data by using native ad-hoc routing
	\ELSIF {(destination is listed in IDRM route table)}
		\STATE tunnel traffic by using IDRM
	\ELSE 
		\STATE drop packet
	\ENDIF	
\end{algorithmic} 
}
\end{algorithm}  

Pseudo code \ref{alg:handleDatapacket} handles data packets
received by IDRM gateways, 
and will be used by both e-IDRM and i-IDRM protocols. 
When a gateway receive a data packet, 
it will first check the destination
appears in the native ad-hoc routing table or not. 
If the destination appears in the native ad-hoc route table,
the gateway will using the native ad-hoc routing to forward the data packet.
If the destination address is not shown in the native ad-hoc route table
but is shown in the IDRM route table,
the gateway will tuunel the data packet to the next gateway according to the
entry in the IDRM route table.
Finally the gateway will drop the data packet if the destination address is not
shown in either of route tables.